\begin{center}
    \large Author Name 3$^{1}$, Author Name 4$^{2}$\\
    \small $^{1}$Institution 3, $^{2}$Institution 4\\
    \today
\end{center}

\begin{abstract}
    This is the abstract for Author 1's paper. It provides a summary of the content, objectives, and results discussed in the paper.
\end{abstract}

\subsection{Introduction}
This section introduces the topic of Author 3's paper and sets up the context for the research or discussion.

\lipsum[1-2] % Generate paragraphs 2 to 4 of Lorem Ipsum

\begin{figure}[H]
\centering
\begin{perlcode}
#!/usr/bin/perl
use strict;
use warnings;

# Print numbers from 1 to 10
print "Numbers from 1 to 10:\n";
for my $i (1 .. 10) {
    print "$i\n";
}

# Open a file for writing
my $filename = 'numbers.txt';
open my $fh, '>', $filename or die "Cannot open file $filename: $!";

# Write numbers to the file
for my $i (1 .. 10) {
    print $fh "$i\n";
}

# Close the file handle
close $fh;

print "Numbers have been written to $filename\n";
\end{perlcode}
\caption{A Perl script that prints numbers from 1 to 10 and writes them to a file. This block an accomodate lines of 110 characters.}
\label{fig:perl_example}
\end{figure}

\subsection{Using Pseudo-code to Describe Algorithms}

We may make use of pseudo-code to describe algorithms that we are discussing or implementing; in many cases, this is better than showing the Perl code for the purpose of simply describing the process or algorithm. Algorithm \ref{alg:find_max} is an example of a beautiful pseudo-code using \LaTeX.

\begin{algorithm}[H] % Use [H] to force placement here
\caption{Find the Maximum Element in an Array}
\label{alg:find_max} % Label for referencing
\begin{algorithmic}[1] % The number indicates line numbers are shown
\Procedure{FindMax}{$A$} % Procedure name and input
    \State $max \gets A[0]$ \Comment{Initialize the maximum to the first element}
    \For{$i \gets 1$ \textbf{to} $length(A) - 1$} \Comment{Loop through the array}
        \If{$A[i] > max$} \Comment{If current element is greater than max}
            \State $max \gets A[i]$ \Comment{Update max}
        \EndIf
    \EndFor
    \State \Return $max$ \Comment{Return the maximum element}
\EndProcedure
\end{algorithmic}
\end{algorithm}

\subsection{Methods}
This section introduces the topic of Author 1's Perl \cite{Wall1996} paper and sets up the context for the research or discussion.

\lipsum[3-4] % Generate paragraphs 2 to 4 of Lorem Ipsum

\subsection{Results}
This section introduces the topic of Author 1's paper and sets up the context for the research or discussion.

\lipsum[5-6] % Generate paragraphs 2 to 4 of Lorem Ipsum


\subsection{Conclusion}
This section introduces the topic of Author 1's paper and sets up the context for the research or discussion.

\lipsum[7-8] % Generate paragraphs 2 to 4 of Lorem Ipsum

\bibliography{references}